\chapter{绪论}

本文在 TeX Live 2015 软件环境下, 使用 XeLaTeX 引擎, 可以顺利编译.
TeX Live 的安装方法, 请移步查看: \url{http://aff.whu.edu.cn/huangzh/}.


引用格式请采用upcite命令行\upcite{AlquezarS96,AlquezarS97}
,而不是cite命令行\cite{AlquezarS96,AlquezarS97}。





\section{研究意义}


{\kaishu 以下文字参看武汉大学图书馆网页:
 \begin{center}
 \url{http://www.lib.whu.edu.cn/web/index.asp?obj_id=457&menu=h}.
 \end{center}
 提交论文之前, 请到该网址查看要求是否有新变化.
}



经研究生院与图书馆共同商议决定, 武汉大学研究生在学位论文答辩通过后,
采用以下方式提交学位论文:首先进行电子版论文网上提交, 经图书馆审核通过后, 进行纸本论文提交.
% \section*{前期准备}

\begin{itemize}
  \item[一、] 请下载《武汉大学学位论文使用授权协议书》(一式两份), 一份论文作者保存, 一份留学校存档.
    留学校存档的协议书事先用钢笔或中性笔填写后,  装订在提交给学校图书馆的纸本论文末页.\footnote{\heiti 特别说明:
    本文前面所附的《武汉大学学位论文使用授权协议书》, 图书馆要求放在最后;
    而研究生院是要求发在文档前面(在论文原创性声明之后). }

    \item[二、]涉密论文缴送到武汉大学档案馆, 由档案馆加盖公章后到图书馆办理相关离校手续.

    \item[三、]电子版论文要求
    

\begin{enumerate}[1.]

  \item 论文的电子文本应采用 Word 或 PDF 格式编辑(不加密).
  \item 电子版全文包括的内容及顺序应与纸本一致: 包括中英文封面、郑重声明、中英文摘要、目录、引言、正文(含图表)、参考文献及附录, 并放在一个文档中.
  \item 电子版文档中不能有空白页、标记、彩色字、乱码.
  \item 目录的页码一定要与正文的章节以及附后的内容相符合. 论文正文页码须从第一页起, 正文之前部分(不包括封面)用罗马字母(I, II, \dots)编页.
\end{enumerate}
\end{itemize}





\begin{figure}[!ht]
\centering
\includegraphics[width=0.5\columnwidth]{data/global_figures/whulogo.pdf}
\caption{
whulogo
}
\label{whulogo}
\end{figure}






进入图书馆主页(\url{http://www.lib.whu.edu.cn})点击``博硕士论文提交'', 进入论文提交系统.
具体步骤和注意事项请参见论文提交过程演示(PPT).
论文提交成功的 3 个工作日后, 可查收 Email 或在``已通过论文名单查询''中查询论文是否审核通过.
如果提交的论文不合格, 请按邮件要求修改后再次进入系统提交论文.



电子版论文提交审核通过的, 请提交 1 份纸本论文到相应的论文纸本缴送地,
末页须装订一份填写好的协议书.



















向武汉大学图书馆提交电子文档, 需单独编译文件: 在文档选项中添加 forlib.

原因: 图书馆要求提交的电子文档不能有空白页、彩色文字.

({\kaishu 这个要求使我在编制模板时遇到了一点问题: 这会导致电子版与纸质版的页码不一致. 博士论文每章的起始页默认在奇数页, 这会不可避免地出现空白页.})


\vfill

本文档下载更新地址: \url{http://aff.whu.edu.cn/huangzh/}. 使用之前, 请移步查看是否有更新.

问题反馈及建议, 请联系: huangzh@whu.edu.cn.



























