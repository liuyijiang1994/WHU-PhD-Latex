% !Mode:: "TeX:UTF-8"
%% 请使用 XeLaTeX 编译本文. 适用于 TeX Live 2015 . 本模板的更新下载地址: http://aff.whu.edu.cn/huangzh/
\documentclass[forlib]{WHUPhd}  % 选项 [forprint]: 交付打印时, 加上此选项, 以消除链接文字之彩色, 避免打印偏淡.
                                      % 选项 [forlib]: 提交给图书馆的电子版, 需要加上选项 forlib, 以消除空白页和彩色链接.
% \usepackage{CJKfntef}
%%%%%%%%%%%%%%%%%%%%%=== 参考文献=== %%%%%%%%%%%%%%%%%%%%%
% \bibliographystyle{abbrv}        % 参考文献样式,  plain,unsrt,alpha,abbrv 等等
% \bibliographystyle{bibdata/gbt7714-2005-plain}
\bibliographystyle{bibdata/gbt7714-numerical}
% \bibliographystyle{bibdata/gbt7714-author-year}
% \bibliographystyle{bibdata/ustcthesis-numerical}
\usepackage[numbers]{bibdata/gbt7714}
\usepackage{natbib}

\usepackage{bm}

\usepackage{multirow}
\usepackage{subfigure}
\usepackage{arydshln}
\usepackage{color}

\usepackage{booktabs}


\usepackage{algorithm}
\usepackage{algorithmic}

\renewcommand{\algorithmicrequire}{ \textbf{输入:}} %Use Input in the format of Algorithm
\renewcommand{\algorithmicensure}{ \textbf{输出:}}



% \usepackage[linesnumbered,ruled]{algorithm2e}



\definecolor{mred}{RGB}{255,114,89}
\definecolor{mblue}{RGB}{102, 102, 255}


\definecolor{blue-1}{RGB}{31,139,191}
\definecolor{red-1}{RGB}{219,93,93}



\newcommand{\myparagraph}[1]{{\vspace{6pt}\noindent\heiti #1\vspace{6pt}}}

\newcommand{\note}[1]{\textcolor{red}{【备注:#1】}}


\setlength{\bibsep}{0.5ex}
%%%%%%%%%%%%%%%%%%%%%%%%%%%%%%%%%%%%%%%%%%%%%%%%%%%%

















\begin{document}
%%%%%%%-------------------------------------------------

\fenleihao{TP309}  % 分类号:《中国图书资料分类法》的类号. 必填. 要根据自己的学科方向填写!!
\miji{公开}                % 密级
\UDC{}               %《国际十进制分类法UDC》的类号. 选填.
\bianhao{10486}  % 学校编号, 10486是武汉大学的编号. 不用改动.

\title{博士论文题目-封面}
\titleh{博士论文题目-header}
\Etitle{Phd thesis title} % 英文题目
\author{xxx}
\Eauthor{Name}            %作者英文名
\Csupervisor{xxx\quad 教授}        %指导教师中文名、职称
\Esupervisor{Prof.~xxx}     %指导教师英文名、职称
\Cmajor{xxx}                   %  学科、专业名称
\Emajor{xxx}% 专业英文名
\Cspeciality{xxxx}                     % 研究方向
\Especiality{xxxx}   % 研究方向
\Schoolname{xxx} %学院英文名. 不确定的话, 请看一下自己学院的网页上是怎么写的. 别搞错了!
\date{二〇二一年五月一日}               %  要注意和英文日期一致!!
\Edate{October 1, 2021}                 % 英文封面日期




%-----------------------------------------------------------------------------
\pdfbookmark[0]{封面}{title}         % 封面页加到 pdf 书签

\maketitle

%-----------------------------------------------------------------------------
% !Mode:: "TeX:UTF-8"

%%% 说明: 此部分需要自己填写的内容:  论文创新点.

%%% ``郑重申明'' 和``论文使用授权协议书'' 分别在 statement.tex 和 protocol.tex 中, 无需改动.

%%%%%%%%%%%%%%%%%%%%%%%%%%%%%
%%% -------------  英文封面 (无需改动)-------------   %%%
%%%%%%%%%%%%%%%%%%%%%%%%%%%%%
\thispagestyle{empty}
\renewcommand{\baselinestretch}{1.5}  %下文的行距
% \vspace*{0.5cm}


\begin{center}{\zihao{3} 
A Dissertation Submitted to \\
Academic Degrees Evaluation Committee of Wuhan University for \\
The Degree of Doctor of Philosophy in Computer Science
\par}\end{center}

\vspace*{2cm}

\begin{center}{\zihao{2}\bf \the\Etitle \par}\end{center}

\vfill

\begin{center}
\zihao{4}
\begin{tabular}{ r l }
 Candidate:      &  {\sc \the\Eauthor}      \\
 Supervisor:     &  {\sc \the\Esupervisor}   \\
 Major:          & \the\Emajor  \\
 Speciality:     & \the\Especiality
\end{tabular}

\vspace*{2cm}
\begin{center}
  \iflib % 向图书馆提交电子文档, 使用黑白校徽.
  \includegraphics[height=4cm]{whu.eps}       %%  黑白的. 很小, 只有 10k.
  \else
  \ifprint % 文档打印, 使用黑白校徽.
  \includegraphics[height=4cm]{whu.eps}       %%  黑白的.
  \else
  \includegraphics[height=4cm]{whulogo.eps} %%  彩色的.
  \fi
  \fi
\end{center}


\zihao{-2}
\the\Schoolname\\
{\sc Wuhan University}

\vspace*{1.0cm}

\the\Edate

\end{center}
%%%%%%%--判断是否需要空白页-----------------------------
  \iflib
  \else
  \newpage
  \thispagestyle{empty}
  \cleardoublepage
  \fi
%%%%%%%-------------------------------------------------
{\pagestyle{empty}
%%%--- 加入``郑重声明'' --- %%%%%%%%%%%%%%%%%
\input{metafile/statement}%%%%%%%%%%%%%%%%%
%%% ---加入``武汉大学学位论文使用授权协议书'' ---  %%%%%%
\input{metafile/protocol}%%%%%%%%%%%%%%%%%%
%%%%%%%%%%%%%%%%%%%%%%%%%%%%%%%
}




%%%%%%%%%%%%%%%%%%%%%%%%%%%%%%%
%%% ------------------- 论文创新点----------------------- %%%
%%%%%%%%%%%%%%%%%%%%%%%%%%%%%%%

\newpage\vspace*{20pt}\thispagestyle{empty}
\begin{center}{\zihao{-2}\heiti 论文创新点}\end{center}
\par\vspace*{30pt}
\baselineskip=20pt



本文的主要创新点包括以下几个部分:
xxxxxxxxxxxxxxxxxxxxx
x

x
x
x
x
x

x
x
x
xxxxxxxxxxxxxx























%%%%%%%%%%%%%%%%%%%%%%%%%%%%%%%
%%%%%%%--请勿删除以下内容 -------------------------------%
%%%%%%%--判断是否需要空白页-----------------------------%
  \iflib
  \let\cleardoublepage\clearpage
  \else
  \newpage
  \thispagestyle{empty}
  \cleardoublepage
  \fi
%%%%%%%---------------------------------------------------%



    % 加入英文封面, 论文创新点, 等等目录之前的部分.
\frontmatter
\pagenumbering{Roman}               % 正文之前的页码用大写罗马字母编号.

\pagestyle{oldplain}\cleardoublepage
\include{metafile/midmatter}      % 加入中英文摘要.

%---把目录加入到书签---%%%%%%%%%%%%%%
\pdfbookmark[0]{目录}{toc}%%%%%%%%%%%%
\pagestyle{oldplain}
\tableofcontents
\pagestyle{oldplain}\cleardoublepage
\newpage  \pagestyle{fancy} \fancyfancy
%------------------------------------------------------------------------------
% \include{metafile/figuresdes}      % 加入图检索.


%------------------------------------------------------------------------------
% \include{metafile/tabledes}      % 加入表检索.




%------------------------------------------------------------------------------



%------------------------------------------------------------------------------
\mainmatter %% 以下是正文
\baselineskip=20pt  % 正文行距为 20 磅
%%%%%%%%%%%%%%%%%%%%%%%%%%%%%%%%%%%%


%%%=====================================================================%%%
\chapter{绪论}

本文在 TeX Live 2015 软件环境下, 使用 XeLaTeX 引擎, 可以顺利编译.
TeX Live 的安装方法, 请移步查看: \url{http://aff.whu.edu.cn/huangzh/}.


引用格式请采用upcite命令行\upcite{AlquezarS96,AlquezarS97}
,而不是cite命令行\cite{AlquezarS96,AlquezarS97}。





\section{研究意义}


{\kaishu 以下文字参看武汉大学图书馆网页:
 \begin{center}
 \url{http://www.lib.whu.edu.cn/web/index.asp?obj_id=457&menu=h}.
 \end{center}
 提交论文之前, 请到该网址查看要求是否有新变化.
}



经研究生院与图书馆共同商议决定, 武汉大学研究生在学位论文答辩通过后,
采用以下方式提交学位论文:首先进行电子版论文网上提交, 经图书馆审核通过后, 进行纸本论文提交.
% \section*{前期准备}

\begin{itemize}
  \item[一、] 请下载《武汉大学学位论文使用授权协议书》(一式两份), 一份论文作者保存, 一份留学校存档.
    留学校存档的协议书事先用钢笔或中性笔填写后,  装订在提交给学校图书馆的纸本论文末页.\footnote{\heiti 特别说明:
    本文前面所附的《武汉大学学位论文使用授权协议书》, 图书馆要求放在最后;
    而研究生院是要求发在文档前面(在论文原创性声明之后). }

    \item[二、]涉密论文缴送到武汉大学档案馆, 由档案馆加盖公章后到图书馆办理相关离校手续.

    \item[三、]电子版论文要求
    

\begin{enumerate}[1.]

  \item 论文的电子文本应采用 Word 或 PDF 格式编辑(不加密).
  \item 电子版全文包括的内容及顺序应与纸本一致: 包括中英文封面、郑重声明、中英文摘要、目录、引言、正文(含图表)、参考文献及附录, 并放在一个文档中.
  \item 电子版文档中不能有空白页、标记、彩色字、乱码.
  \item 目录的页码一定要与正文的章节以及附后的内容相符合. 论文正文页码须从第一页起, 正文之前部分(不包括封面)用罗马字母(I, II, \dots)编页.
\end{enumerate}
\end{itemize}





\begin{figure}[!ht]
\centering
\includegraphics[width=0.5\columnwidth]{data/global_figures/whulogo.pdf}
\caption{
whulogo
}
\label{whulogo}
\end{figure}






进入图书馆主页(\url{http://www.lib.whu.edu.cn})点击``博硕士论文提交'', 进入论文提交系统.
具体步骤和注意事项请参见论文提交过程演示(PPT).
论文提交成功的 3 个工作日后, 可查收 Email 或在``已通过论文名单查询''中查询论文是否审核通过.
如果提交的论文不合格, 请按邮件要求修改后再次进入系统提交论文.



电子版论文提交审核通过的, 请提交 1 份纸本论文到相应的论文纸本缴送地,
末页须装订一份填写好的协议书.



















向武汉大学图书馆提交电子文档, 需单独编译文件: 在文档选项中添加 forlib.

原因: 图书馆要求提交的电子文档不能有空白页、彩色文字.

({\kaishu 这个要求使我在编制模板时遇到了一点问题: 这会导致电子版与纸质版的页码不一致. 博士论文每章的起始页默认在奇数页, 这会不可避免地出现空白页.})


\vfill

本文档下载更新地址: \url{http://aff.whu.edu.cn/huangzh/}. 使用之前, 请移步查看是否有更新.

问题反馈及建议, 请联系: huangzh@whu.edu.cn.



























  


%%%=====================================================================%%%

\chapter{第二章}


\begin{itemize}
  \item[一、] 请下载《武汉大学学位论文使用授权协议书》(一式两份), 一份论文作者保存, 一份留学校存档.
    留学校存档的协议书事先用钢笔或中性笔填写后,  装订在提交给学校图书馆的纸本论文末页.\footnote{\heiti 特别说明:
    本文前面所附的《武汉大学学位论文使用授权协议书》, 图书馆要求放在最后;
    而研究生院是要求发在文档前面(在论文原创性声明之后). }

    \item[二、]涉密论文缴送到武汉大学档案馆, 由档案馆加盖公章后到图书馆办理相关离校手续.

    \item[三、]电子版论文要求
    

\begin{enumerate}[1.]

  \item 论文的电子文本应采用 Word 或 PDF 格式编辑(不加密).
  \item 电子版全文包括的内容及顺序应与纸本一致: 包括中英文封面、郑重声明、中英文摘要、目录、引言、正文(含图表)、参考文献及附录, 并放在一个文档中.
  \item 电子版文档中不能有空白页、标记、彩色字、乱码.
  \item 目录的页码一定要与正文的章节以及附后的内容相符合. 论文正文页码须从第一页起, 正文之前部分(不包括封面)用罗马字母(I, II, \dots)编页.
\end{enumerate}
\end{itemize}





   
  

%%%=====================================================================%%%

%%%=====================================================================%%%
\chapter{总结与展望}


\section{全文工作总结}
{\kaishu 以下文字参看武汉大学图书馆网页:
 \begin{center}
 \url{http://www.lib.whu.edu.cn/web/index.asp?obj_id=457&menu=h}.
 \end{center}
 提交论文之前, 请到该网址查看要求是否有新变化.
}

    经研究生院与图书馆共同商议决定, 武汉大学研究生在学位论文答辩通过后,
    采用以下方式提交学位论文:首先进行电子版论文网上提交, 经图书馆审核通过后, 进行纸本论文提交.

% \section*{前期准备}

\begin{itemize}
  \item[一、] 请下载《武汉大学学位论文使用授权协议书》(一式两份), 一份论文作者保存, 一份留学校存档.
    留学校存档的协议书事先用钢笔或中性笔填写后,  装订在提交给学校图书馆的纸本论文末页.\footnote{\heiti 特别说明:
    本文前面所附的《武汉大学学位论文使用授权协议书》, 图书馆要求放在最后;
    而研究生院是要求发在文档前面(在论文原创性声明之后). }

    \item[二、]涉密论文缴送到武汉大学档案馆, 由档案馆加盖公章后到图书馆办理相关离校手续.

    \item[三、]电子版论文要求
    

\begin{enumerate}[1.]

  \item 论文的电子文本应采用 Word 或 PDF 格式编辑(不加密).
  \item 电子版全文包括的内容及顺序应与纸本一致: 包括中英文封面、郑重声明、中英文摘要、目录、引言、正文(含图表)、参考文献及附录, 并放在一个文档中.
  \item 电子版文档中不能有空白页、标记、彩色字、乱码.
  \item 目录的页码一定要与正文的章节以及附后的内容相符合. 论文正文页码须从第一页起, 正文之前部分(不包括封面)用罗马字母(I, II, \dots)编页.
\end{enumerate}
\end{itemize}








\section{未来研究展望}

    进入图书馆主页(\url{http://www.lib.whu.edu.cn})点击``博硕士论文提交'', 进入论文提交系统.
    具体步骤和注意事项请参见论文提交过程演示(PPT).
    论文提交成功的 3 个工作日后, 可查收 Email 或在``已通过论文名单查询''中查询论文是否审核通过.
    如果提交的论文不合格, 请按邮件要求修改后再次进入系统提交论文.



    电子版论文提交审核通过的, 请提交 1 份纸本论文到相应的论文纸本缴送地,
    末页须装订一份填写好的协议书.






   

%%%=====================================================================%%%




%%%------------附录. 不需要的话, 可以删除.--------
% \appendix

% \chapter{测试1}

% 测试

% \chapter{测试2}

% 测试

% \chapter{测试3}

% 测试




%%%%%%%%%%%%%%%%%=== 参考文献 ==%%%%%%%%%%%%%%%%%%%
\cleardoublepage\phantomsection

\addcontentsline{toc}{chapter}{参考文献}

{\zihao{5}
\bibliography{bibdata/bibtex}
% \bibliography{data/sec1/sec1}
}



\backmatter

\include{metafile/backmatter} %%%致谢、作品列表 .

\cleardoublepage
\end{document}

